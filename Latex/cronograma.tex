\documentclass[border=10pt]{standalone}%{article}
%\usepackage[margin=1cm]{geometry}
\usepackage{pgfgantt}
\usepackage{graphicx}
\usepackage{xcolor}
%\usetikzlibrary{positioning}

\ganttset{group/.append style={orange},
milestone/.append style={red},
progress label node anchor/.append style={text=red}}

\begin{document}

%    \begin{figure}
%    \centering
     \begin{ganttchart}[%Specs
     y unit title=0.6cm,
     y unit chart=0.9cm,
     vgrid,hgrid,
     title height=1,
%     title/.style={fill=none},
     title label font=\bfseries\footnotesize,
     bar/.style={fill=blue},
     bar height=0.7,
%   progress label text={},
     group right shift=0,
     group top shift=0.7,
     group height=.3,
     group peaks width={0.2},
     inline]{1}{48}
    %labels
    \gantttitle{T e s i s}{48}\\  % title 1
    \gantttitle[]{Enero-Junio}{24}                 % title 2
    \gantttitle[]{Agosto-Diciembre}{24} \\              
    \gantttitle{Enero}{4}                      % title 3
    \gantttitle{Febrero}{4}
    \gantttitle{Marzo}{4}
    \gantttitle{Abril}{4}
    \gantttitle{Mayo}{4}
    \gantttitle{Junio}{4}
    \gantttitle{Julio}{4} 
    \gantttitle{Agosto}{4}
    \gantttitle{Septiembre}{4}
    \gantttitle{Octubre}{4}
    \gantttitle{Noviembre}{4}
    \gantttitle{Diciembre}{4}
    \\
    % Setting group if any
    \ganttgroup[inline=false]{Group 1}{1}{5}\\ 
    \ganttbar[progress=90,inline=false]{Planning}{1}{5}\\
    \ganttmilestone[inline=false]{Planeación}{5} \\

%    \ganttbar[progress=50,inline=false, bar progress label node/.append style={below left= 10pt and 7pt}]{Task B}{20}{24} \\ \\
%    \ganttbar[progress=30,inline=false]{Task C}{20}{26}\\ 


    \ganttbar[progress=0,inline=false]{Expo Semi}{12}{16} \\ 
    \ganttmilestone[inline=false]{Expo Semi v1}{14} \\


    \ganttgroup[inline=false]{Parte Experimental}{6}{16}\\ 
%reproducir paper de Suzuki con oculus
%	pulir lente del oculus ~1 día
%	conectar ekg a oculus ~1 día
%	probar latencia por server para conectar el ekg por usb a la pc ~2 dias
%	crear vision "monoscópica" en oculus  ~1 día
%	hacer shader del parpadeo de las manos ~1 día
%	hacer interfaz para el cuestionario en oculus ~3 días
%	usar sqlite pa guardar los datos y respuestas ~3 días
    \ganttbar[progress=75,inline=false]{Experimento Suzuki}{6}{11} \\ 

%planear piloto
%	agregar cambio de espectroscópico a "monoscópico" ~1 día
%	calcular tiempos de la sesión basado en el paper de Suzuki ~1 día
%	ver tiempo y participantes totales basado en el tiempo de entrega
%	conseguir participantes y agendarlos (no sé si este va aquí o en la aplicación)
%	ver si la asignación aleatoria de espectroscópico o monoscópico se hace directamente en el oculus
%		de ser así programarla
%	agregar variable monoscópica o espectroscópica al esquema de la base de datos
%Aplicar piloto
%	basado en la agenda de participantes, aplicar y tomar los datos
%Análisis de los datos
%	falta ver qué análisis voy a hacer
    \ganttbar[progress=0,inline=false]{Piloto}{11}{16} \\ 




    \ganttgroup[inline=false]{Parte Teórica}{6}{16}\\ 
    \ganttbar[progress=60,inline=false]{Lectura estereopsis}{6}{6} \\ 
%1 Introducción
%		1.1 Planteación del Problema
%			falta todo 1~2 p * 150 min = 300m ~1 día
%		1.2 Justificación
%			falta todo 1~2 p * 150 min = 300m ~1 día
%		1.3 ObjetivosT e s i sT e s i s
%			falta todo 1~2 p * 150 min = 300m ~1 día
%		1.4 Hipótesis
%			falta todo 1~2 p  * 150 min = 300m ~1 día
    \ganttbar[progress=30,inline=false]{Capítulo 1}{7}{7} \\ 

%2 Marco teórico
%		2.1.1 La experiencia de pertenencia del cuerpo
%			1~2 p * 150 min = 300m ~1 día
%		2.1.2 Fusión multi-modal
%			2~3 p * 150 min = 450m ~2 días
%		2.1.3 Información interoceptiva
%			2~3 p * 150 min = 450m ~2 días
%		2.2 La ilusión de la mano de hule
%			~5 p * 150 min = 750m ~3 días
%		2.2.1 Medidas de la ilusión
%			falta todo 3~ 4 p * 150 min = 600m = 2d
%		2.2.2 Ilusión + interocepción
%			2~3 p * 150 min = 450m ~2 días
%		2.3 Estado del arte
%			~8 p * 150 min = 1200m ~4 días
% 16 dias ~ 3 semanas
    \ganttbar[progress=15,inline=false]{Capítulo 2}{10}{12} \\ 

%3 Metodología
%       falta todo
    \ganttbar[progress=0,inline=false]{Capítulo 3}{13}{16} \\ 

    \ganttmilestone[inline=false]{Escrito Pre-Res v1}{16} \\




    \ganttgroup[inline=false]{Expo}{13}{17}\\ 
    \ganttbar[progress=0,inline=false]{Expo Pre-Res}{13}{17} \\ 
    \ganttmilestone[inline=false]{Expo Pre-Res v1}{17} \\



    \ganttgroup[inline=false]{Correciones Escrito Pre-Res}{17}{20}\\ 

    \ganttbar[progress=0, inline=false]{1ra Revisión Escrito Pre-Res}{17}{17} \\
    \ganttbar[progress=0, inline=false]{1ra Corrección Escrito Pre-Res}{18}{18} \\
    \ganttbar[progress=0, inline=false]{2da Revisión Escrito Pre-Res}{19}{19} \\
    \ganttbar[progress=0, inline=false]{2da Corrección Escrito Pre-Res}{20}{20} \\


    \ganttgroup[inline=false]{Correciones Expo Pre-Res}{18}{21}\\ 

    \ganttbar[progress=0, inline=false]{1ra Revisión Expo Pre-Res}{18}{18} \\
    \ganttbar[progress=0, inline=false]{1ra Corrección Expo Pre-Res}{19}{19} \\
    \ganttbar[progress=0, inline=false]{2da Revisión Expo Pre-Res}{20}{20} \\
    \ganttbar[progress=0, inline=false]{2da Corrección Expo Pre-Res}{21}{21} \\

    \ganttbar[progress=0, inline=false]{Examen Pre-Residencia}{22}{22} \\





% Residensia (Julio - Diciembre)

    \ganttmilestone[inline=false]{Escrito Res v1}{39} \\
    \ganttgroup[inline=false]{Correciones Escrito Res}{40}{43}\\ 

    \ganttbar[progress=0, inline=false]{1ra Revisión Escrito Res}{40}{40} \\
    \ganttbar[progress=0, inline=false]{1ra Corrección Escrito Res}{41}{41} \\
    \ganttbar[progress=0, inline=false]{2da Revisión Escrito Res}{42}{42} \\
    \ganttbar[progress=0, inline=false]{2da Corrección Escrito Res}{43}{43} \\


    \ganttmilestone[inline=false]{Expo Res v1}{40} \\
    \ganttgroup[inline=false]{Correciones Escrito Res}{41}{44}\\ 

    \ganttbar[progress=0, inline=false]{1ra Revisión Expo Res}{41}{41} \\
    \ganttbar[progress=0, inline=false]{1ra Corrección Expo Res}{42}{42} \\
    \ganttbar[progress=0, inline=false]{2da Revisión Expo Res}{43}{43} \\
    \ganttbar[progress=0, inline=false]{2da Corrección Expo Res}{44}{44} \\

    \ganttbar[progress=0, inline=false]{Examen Residencia}{45}{45} \\
    
\end{ganttchart}
%    \caption{Gantt diagram for 2013--2014 Project}
%\end{figure}
\end{document}